\documentclass[11pt]{article}
\usepackage{enumerate}
\usepackage{amsmath}
\usepackage{blindtext}
\usepackage{multicol}
\usepackage{amsfonts}
\usepackage[left=1.5cm,top=2cm,right=1.5cm,bottom=2cm]{geometry}

\newcommand*\varhrulefill[1][0.4pt]{\leavevmode\leaders\hrule height#1\hfill\kern0pt}
\begin{document}

\begin{center}
    \textbf{MATEMÁTICA}
\end{center}

\noindent\varhrulefill[0.4mm]

\vspace{6pt}

\noindent \textbf{Notações}

\vspace{6pt}

$\mathbb{N}\quad = \{ 1,2,3, \dots \}$: o conjunto dos números naturais.

$\mathbb{R}\quad$: o conjunto dos números reais.

$\mathbb{C}\quad$: o conjunto dos números complexos.

$i \, \, \quad$: unidade imaginária, $i^2 = -1$.

\vspace{6pt}

\noindent Observação: Os sistemas de coordenadas considerados são os cartesianos retangulares.

\noindent\varhrulefill[0.4mm]

\vspace{6pt}

\parindent0em

\textbf{Questão 1.} Qual o resto da divisão por 3 do determinante:
\[
\begin{vmatrix}
4 & 1 & 3 & -6\\
(3-4) & (6-1) & (-3-5) & (9+6)\\
5 & 1 & 2 & 3\\
1 & 1 & 2 & 5
\end{vmatrix}
\]

\begin{multicols}{5}
    \begin{enumerate}[\bf A (\quad)]
        \item 0 
        \item 3
        \item 7
        \item 1
        \item n.d.a.
    \end{enumerate}
\end{multicols}

\textbf{Questão 2.} Sejam $\alpha$ e $\beta$ planos não paralelos interceptados ortogonalmente pelo plano $\gamma$. Sejam $r$, $s$ e $t$ respectivamente as interseções de $\alpha$ e $\beta$, $\alpha$ e $\gamma$ e $\beta$ e $\gamma$. Qual das afirmações abaixo é sempre correta? 

\begin{enumerate}[\bf A (\quad)]
    \item $r$, $s$ e $t$ formam oito triedros tri-retângulos
    \item Existe um ponto $P$ de $r$ tal que, qualquer reta de $\gamma$ que passa por $P$ é ortogonal a $r$
    \item $r$ pode não interceptar $\gamma$
    \item $t$ é perpendicular a $\alpha$
    \item Nenhuma dessas afirmações é correta
\end{enumerate}

\textbf{Questão 3.} O produto dos termos da seguinte P.G. ($-\sqrt{3}$, $3$, $-3\sqrt{3}$, ..., $-81\sqrt{3}$) é:

\begin{multicols}{3}
    \begin{enumerate}[\bf A (\quad)]
        \item $-\sqrt{3^{25}}$
        \item $-\sqrt{3^{42}}$
        \item $-\sqrt{5.3^9}$
        \item $-\sqrt{3^{45}}$
        \item n.d.a.
    \end{enumerate}
\end{multicols}

\textbf{Questão 4.} Se $f$ é uma função real de variável real dada por $f(x) = x^2$, então $f(x^2 + y^2)$ é igual a:

\begin{enumerate}[\bf A (\quad)]
    \item $f(f(x)) + f(y) + 2f(x)f(y)$ para todo $x$ e $y$
    \item $f(x^2) + 2f(f(x)) + f(x)f(y)$ para todo $x$ e $y$
    \item $f(x^2) + f(y^2) + f(x)f(y)$ para todo $x$ e $y$
    \item $f(f(x)) + f(f(y)) + 2f(x)f(y)$ para todo $x$ e $y$
    \item $f(f(x)) + 2f(y^2) + 2f(x)f(y)$ para todo $x$ e $y$
\end{enumerate}

\textbf{Questão 5.} Uma solução da equação: $24x^5 -4x^4 + 49x^3 - 2x^2 + x -29 = 0$ é:

\begin{multicols}{3}
    \begin{enumerate}[\bf A (\quad)]
        \item $x = 2/3$
        \item $x = 11/12$
        \item $x = 3/4$
        \item $x = 4/3$
        \item n.d.a.
    \end{enumerate}
\end{multicols}

\textbf{Questão 6.} Seja a desigualdade: $2(\log x)^2 - \log x > 6$. Determinando as soluções desta inequação obtemos:

\begin{enumerate}[\bf A (\quad)]
    \item $0 < x < 1/e$ e $x > 10^2$
    \item $0 < x < e^{-3/2}$ e $x > e^2$
    \item $0 < x < e$ e $x < 10$
    \item $1/e < x < 1$ e $x > e$
    \item n.d.a.
\end{enumerate}

\textbf{Questão 7.} Dada uma circunferência de diâmetro $AB$, centro $O$ e um ponto $C$ da circunferência, achar o lugar geométrico dos pontos de intersecção do raio $OC$ a paralela ao diâmetro $AB$ e passando pelo pé da perpendicular a $AC$ tirada por $O$.

\begin{enumerate}[\bf A (\quad)]
    \item um segmento de reta paralelo a $AB$.
    \item uma circunferência de raio $2R/3$ e origem $O$.
    \item uma circunferência de raio $R/2$ e origem $O$.
    \item uma elipse de semi-eixo maior $OA$.
    \item n.d.a.
\end{enumerate}

\textbf{Questão 8.} Consideremos a equação:

$$
\{ \log(\sin x)\}^2 - \log(\sin x ) - 6 = 0
$$

A(s) solução(es) da equação acima é dada por:

\begin{enumerate}[\bf A (\quad)]
    \item $x = \arcsin (e^2)$ e $x = \arcsin (3)$
    \item $x = \arcsin (1/2)$ e $x = \arcsin (1/3)$
    \item $x = \arctan (e^2)$ e $x = \arccos (3)$
    \item $x = \arcsin (1/e^2)$
    \item n.d.a.
\end{enumerate}

\textbf{Questão 9.} Uma progressão geométrica de 3 termos positivos cuja soma é $m$ tem seu segundo termo igual a 1. Que valores devem assumir $m$, para que o problema tenha solução?

\begin{multicols}{3}
    \begin{enumerate}[\bf A (\quad)]
        \item $0 < m \leq 1$
        \item $1 \leq m < 3$
        \item $m \geq 3$
        \item $1 \leq m \leq 2$
        \item n.d.a.
    \end{enumerate}
\end{multicols}

\textbf{Questão 10.} Dada a equação, $\log (\cos x) = \tan x $, as soluções desta equação em $x$ satisfazem a relação:

\begin{multicols}{3}
    \begin{enumerate}[\bf A (\quad)]
        \item $3\pi / 2 < x \leq 2$
        \item $0 < x < \pi /2$
        \item $0 < x < \pi$
        \item $- \pi /2 < x < \pi /2$
        \item n.d.a.
    \end{enumerate}
\end{multicols}

\textbf{Questão 11.} Dado um cone reto de geratriz $g$ e altura $h$, calcular a que distância do vértice deveremos passar um plano paralelo à base, a fim de que a secção obtida seja equivalente à área lateral do tronco formado.

\begin{multicols}{2}
    \begin{enumerate}[\bf A (\quad)]
        \item $\sqrt{g(g+h)}$
        \item $\sqrt{g(g - \sqrt{g^2 - h^2})}$
        \item $\sqrt{g^2 - \sqrt{g^2 - h^2}}$
        \item $\sqrt{h^2 - g \sqrt{g^2 - h^2}}$
        \item n.d.a.
    \end{enumerate}
\end{multicols}

\textbf{Questão 12.} Dado o sistema de desigualdades ($a > 0$, $b > 0$, $b \neq a$).

$$
\begin{cases}
ax + bx \geq 0 \\
\dfrac{a}{4}x^2 - bx + (2b-a) < 0 
\end{cases}
$$

\begin{multicols}{3}
    \begin{enumerate}[\bf A (\quad)]
        \item $x < -b/a$ e $b > a$
        \item $x > 2$ e $b < a$
        \item $0 < x < 1$ e $b > 3a/4$
        \item $x > 4b/a-2$ e $a > 2b$
        \item n.d.a.
    \end{enumerate}
\end{multicols}


\textbf{Questão 13.} A seguinte soma: $\log \dfrac{1}{2} + \log \dfrac{1}{4} + ... + \log \dfrac{1}{2^n}$, com $n$ natural, é igual a:

\begin{enumerate}[\bf A (\quad)]
    \item $\log \dfrac{n + n^3}{2}$
    \item $(n + n^2) \log \sqrt{\dfrac{1}{2}}$
    \item $-n(n+1) \log 2$
    \item $\left( \dfrac{n^2 - 1}{2} \right) 2^{\sqrt{2}}$
    \item n.d.a.
\end{enumerate}


\textbf{Questão 14.} Qual o resto da divisão por $(x - a)$, do polinômio:

\[
\begin{vmatrix}
1 & x & x^2 & x^3\\
1 & a & a^2 & a^3\\
1 & b & b^2 & b^3\\
1 & c & c^2 & c^3
\end{vmatrix}
\]

\begin{multicols}{3}
    \begin{enumerate}[\bf A (\quad)]
        \item $2x^3 + c$
        \item $6X^2 + 7$
        \item $5$
        \item $0$
        \item n.d.a.
    \end{enumerate}
\end{multicols}

\textbf{Questão 15.} Dividindo o polinômio: $P(x) = x^3 + x^2 + x + 1$ pelo polinômio $Q(x)$ obtemos o quociente $S(x) = 1 + x$ e o resto $R(x) = x + 1$. O polinômio $Q(x)$ satisfaz:

\begin{multicols}{3}
    \begin{enumerate}[\bf A (\quad)]
        \item $Q(2) = 0$
        \item $Q(3) = 0$
        \item $Q(0) \neq 0$
        \item $Q(1) \neq 0$
        \item n.d.a.
    \end{enumerate}
\end{multicols}

\textbf{Questão 16.} Seja $P(x) = a_0 + a_1x + a_2 x^2 + a_3 x^3 + ... + a_{100} x^{100}$, onde $x^{100} = 1$, um polinômio divisível por $(x + 9)^{100}$. Nestas condições temos:

\begin{multicols}{3}
    \begin{enumerate}[\bf A (\quad)]
        \item $a_2 = 50.99.98$
        \item $a_2 = 100!/(2!98!)$
        \item $a_2 = 99!/(2!98!)$
        \item $(100!9^2)/(2!98!)$
        \item n.d.a.
    \end{enumerate}
\end{multicols}

\textbf{Questão 17.} Determinando-se a condição sobre $t$ para que a equação: $4^x - (\log t + 3)2^x - \log t = 0$. Admita duas raízes reais e distintas, obtemos:

\begin{multicols}{3}
    \begin{enumerate}[\bf A (\quad)]
        \item $e^{-3} \leq t \leq 1$
        \item $t \geq 0$
        \item $e^{-3} < t < 1$
        \item $3 < t < e^2$
        \item n.d.a.
    \end{enumerate}
\end{multicols}

\textbf{Questão 18.} Qual é o menor valor de $x$ que verifica a equação: $\tan x + 3 \cot x = 3$?


\begin{enumerate}[\bf A (\quad)]
    \item $x = \pi / 4$
    \item para todo $x$ e $(0, \pi/2)$
    \item para nenhum valor de $x$
    \item para todo valor de $x \neq n \pi / 2$ onde $n = 0, \pm 1, \pm 2, ...$
    \item apenas para $x$ no $3^{\circ}$ quadrante
\end{enumerate}

\textbf{Questão 19.} Dispomos de seis cores diferentes. Cada face de um cubo será pintada com uma cor diferente, de forma que as seis cores sejam utilizadas. De quantas formas diferentes isto pode ser feito, se uma maneira é considerada idêntica a outra desde que possa ser obtida por rotação do cubo?

\begin{multicols}{3}
    \begin{enumerate}[\bf A (\quad)]
        \item 30
        \item 12
        \item 36
        \item 18
        \item n.d.a.
    \end{enumerate}
\end{multicols}

\textbf{Questão 20.} A igualdade $\dfrac{\cos x}{2} = \cos \dfrac{x}{2}$ é verificada para

\begin{enumerate}[\bf A (\quad)]
    \item para qualquer valor de $x$
    \item para qualquer valor de $x \neq n \pi / 2$ onde $n = 0, \pm 1, \pm 2, ...$
    \item para $x > 2 \arccos \left( \dfrac{1 - \sqrt{3}}{2} \right)$
    \item para nenhum valor de $x$
    \item para $x = 2\arccos ( \cos 60^{\circ} - \cos 30^{\circ})$
\end{enumerate}

\textbf{Questão 21.} Cortando-se determinado prisma triangular, reto, por um plano $\alpha$ que forma um ângulo de $45^{\circ}$ com o plano da base $ABC$ observamos que a reta $r$, interseção de $\alpha$ com o plano da base, dista 7cm de $A$, 5cm de $B$ e 2cm de $C$. Se área da face for 21cm², o volume do tronco  de prisma compreendido entre a base $ABC$ e o plano $\alpha$ será:

\begin{multicols}{3}
    \begin{enumerate}[\bf A (\quad)]
        \item $105 \, cm^3$
        \item $294 \, cm^3$
        \item $98 \, cm^3$
        \item $98 \sqrt{2} \, cm^3$
        \item $98 /\sqrt{2} \, cm^3$
    \end{enumerate}
\end{multicols}

\textbf{Questão 23.} Seja $n$ um número inteiro $n \geq  1$ e $x \in (0, \pi/2)$. Qual das afirmações abaixo é sempre verdadeira? 

\begin{enumerate}[\bf A (\quad)]
    \item $(1 - \sin x)^n \geq 1 - n \sin x$
    \item $(1 - \sin x)^n \geq 1 - n \sin x$ para apenas $n$ par
    \item $(1 - \sin x)^n \leq 1 - n \sin x$
    \item $(1 - \sin x)^n \leq 1 - n \cos x$
    \item n.d.a.
\end{enumerate}


\textbf{Questão 24.} Seja $x \in (0, \pi/2)$. Qual afirmação abaixo é verdadeira?

\begin{multicols}{3}
    \begin{enumerate}[\bf A (\quad)]
        \item $\dfrac{\sin x}{\cos x} + \dfrac{\cos x}{\sin x} \leq 1$
        \item $\dfrac{\sin x}{\cos x} + \dfrac{\cos x}{\sin x} \leq 2$
        \item $\dfrac{\sin x}{\cos x} + \dfrac{\cos x}{\sin x} \geq 1$
        \item $\dfrac{\sin x}{\cos x} + \dfrac{\cos x}{\sin x} = 2$
        \item n.d.a.
    \end{enumerate}
\end{multicols}

\textbf{Questão 25.} Qual é o maior número de partes em que um plano pode ser dividido por $n$ linhas retas? 

\begin{multicols}{3}
    \begin{enumerate}[\bf A (\quad)]
        \item $n^2$
        \item $n(n+1)$
        \item $n(n+1)/2$
        \item $(n^2 + n +2)/2$
        \item n.d.a.
    \end{enumerate}
\end{multicols}


\end{document}