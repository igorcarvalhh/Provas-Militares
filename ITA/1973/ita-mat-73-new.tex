\documentclass[11pt]{article}
\usepackage{enumerate}
\usepackage{amsmath}
\usepackage{blindtext}
\usepackage{multicol}
\usepackage{amsfonts}
\usepackage[left=1.5cm,top=2cm,right=1.5cm,bottom=2cm]{geometry}

\newcommand*\varhrulefill[1][0.4pt]{\leavevmode\leaders\hrule height#1\hfill\kern0pt}
\begin{document}

\begin{center}
    \textbf{MATEMÁTICA}
\end{center}

\noindent\varhrulefill[0.4mm]

\vspace{6pt}

\noindent \textbf{Notações}

\vspace{6pt}

$\mathbb{N}\quad = \{ 1,2,3, \dots \}$: o conjunto dos números naturais.

$\mathbb{R}\quad$: o conjunto dos números reais.

$\mathbb{C}\quad$: o conjunto dos números complexos.

$i \, \, \quad$: unidade imaginária, $i^2 = -1$.

\vspace{6pt}

\noindent Observação: Os sistemas de coordenadas considerados são os cartesianos retangulares.

\noindent\varhrulefill[0.4mm]

\vspace{6pt}

\parindent0em

\textbf{Questão 1.} Sabe-se que a média harmónica entre o raio e a altura de  um  cilindro  de  revolução  vale  4.  Quanto  valerá  a  relação do volume para a área total deste cilindro? 

\begin{multicols}{5}
    \begin{enumerate}[\bf A (\quad)]
        \item 1 
        \item 2
        \item 2,5
        \item 3
        \item n.d.a.
    \end{enumerate}
\end{multicols}

\textbf{Questão 2.} O  crescimento  de  uma  certa  cultura  de  bactérias  obedece  à  função  $X(t)  =  Ce^{kt}$,  onde  $X(t)$  é  o  número  de  bactérias no tempo $t \geq 0$ ; $C$ e $k$ são constantes positivas, (e é a base do logaritmo neperiano). Verificando-se que o número  inicial  de  bactérias  $X(0)$,  duplica  em  4  horas,  quantas bactérias se pode esperar no fim de 6 horas?

\begin{enumerate}[\bf A (\quad)]
    \item 3 vezes o número inicial
    \item 2,5 vezes o número inicial
    \item $2\sqrt{2}$ vezes o número inicial
    \item $2\sqrt[3]{2}$ vezes o número inicial
    \item n.d.a.
\end{enumerate}

\textbf{Questão 3.} Um    navio,    navegando    em    linha    reta,    passa    sucessivamente  pelos  pontos  $A$,  $B$  e  $C$.  O  comandante  quando  o  navio  está  em  $A$,  observa  um  farol  em  $L$,  e  calcula  o  ângulo  $C\hat{A}L=  30^{\circ}$.  Após  navegar  4  milhas  até  $B$, verifica o ângulo $C\hat{B}L = 75^{\circ}$. Quantas milhas separam o farol do ponto $B$?

\begin{multicols}{5}
    \begin{enumerate}[\bf A (\quad)]
        \item 4
        \item $2\sqrt{2}$
        \item 8/3
        \item $\sqrt{3}/2$
        \item n.d.a.
    \end{enumerate}
\end{multicols}

\textbf{Questão 4.} Consideremos um cone de revolução de altura $h$, e um cilindro  nele  inscrito.  Seja  $d$  a  distância  do  vértice  do  cone  à  base  superior  do  cilindro.  A  altura  $H$  de  um  segundo   cilindro   inscrito   neste   cone   (diferente   do   primeiro) e de mesmo volume do primeiro é dada por: 

\begin{enumerate}[\bf A (\quad)]
    \item $H = \dfrac{h - \sqrt{h - d}}{3}$
    \item $H = \dfrac{h \pm \sqrt{h^2 - d^2}}{3}$
    \item $H = \dfrac{h - d + h\sqrt{h^2 - d^2}}{2}$
    \item $H = \dfrac{h + d - \sqrt{(h - d)(h + 3d)}}{2}$
    \item n.d.a.
\end{enumerate}

\textbf{Questão 5.} O coeficiente de $a^{n-1-p}b^p$ no produto de:

$$
a^k + \binom{k}{1}a^{k-1}b + \dots + \binom{k}{p}a^{k-p}b^p + \dots + b^k
$$

por $(a + b)$, se $k = n$, vale:

\begin{multicols}{3}
    \begin{enumerate}[\bf A (\quad)]
        \item $\binom{n}{p}$
        \item $\binom{n + 1}{p}$
        \item $\binom{n - 1}{p}$
        \item $\binom{n + 1}{p + 1}$
        \item n.d.a.
    \end{enumerate}
\end{multicols}

\textbf{Questão 6.} A desigualdade $\sqrt[x-3]{x} \sqrt{x} \leq 1/x$ é válida para:

\begin{enumerate}[\bf A (\quad)]
    \item qualquer $x$ positivo
    \item $1 \leq x \leq 3$
    \item $0 < x \leq 1$ ou $2 \leq x \leq 3$
    \item $0 < x \leq 1$ ou $2 \leq x < 3$
    \item n.d.a.
\end{enumerate}

\textbf{Questão 7.} Suponhamos que $p$ e $q$ são os catetos de um triângulo retângulo  e  $h$  a  altura  relativa  à  hipotenusa  do  mesmo.  Nestas condições, podemos afirmar que a equação:

$$
\frac{2}{p} x^2 - \frac{2}{h} x + \frac{1}{q} = 0
$$

\begin{enumerate}[\bf A (\quad)]
    \item não admite soluções reais
    \item admite uma raiz da forma $m\sqrt{-1}$, onde $m \in \mathbb{R}$, $m > 0$
    \item admite sempre raízes reais
    \item admite uma raiz da forma $-m\sqrt{-1}$ , onde $m \in \mathbb{R}$, $m > 0$
    \item n.d.a.
\end{enumerate}

\textbf{Questão 8.} A respeito da equação: 

$$
3x^2 - 4x + \sqrt{3x^2 - 4x - 6} = 18
$$

podemos dizer:

\begin{enumerate}[\bf A (\quad)]
    \item $\dfrac{2 \pm \sqrt{10}}{3}$ são raízes 
    \item a única raiz real é $x = 3$
    \item a única raiz real é $x = 2 + \sqrt{10}$
    \item tem duas raízes reais e imaginárias 
    \item n.d.a.
\end{enumerate}

\textbf{Questão 9.} A base $AB$, de uma folha de papel triangular que está sobre   uma   mesa,   mede   12   cm.   O   papel   é   dobrado   levantando-se  sua  base,  de  modo  que  a  dobra  fique  paralela  à  mesma.  A  área  da  parte  do  triângulo  que  fica  visível após o papel ter sido dobrado, vale 0,30 da área do triângulo $ABC$. O comprimento da dobra vale: 

\begin{multicols}{3}
    \begin{enumerate}[\bf A (\quad)]
        \item 9,6 cm
        \item 9,4 cm
        \item 10 cm
        \item 8 cm
        \item n.d.a.
    \end{enumerate}
\end{multicols}

\textbf{Questão 10.} Os valores de $x$ que verificam a desigualdade: 

$$
\frac{1}{\log_e x} + \frac{1}{\log_x e-1} > 1
$$

\begin{multicols}{3}
    \begin{enumerate}[\bf A (\quad)]
        \item $x > 1$
        \item $x > e2$
        \item $0 < x < e$
        \item $1 < x < e$
        \item n.d.a.
    \end{enumerate}
\end{multicols}

\textbf{Questão 11.} Sejam $n \in \mathbb{N}^*$, $p \in \mathbb{N}$, onde $\mathbb{N} = \{ 0, 1, 2, 3, \dots \}$ e $\mathbb{N}^* = \{ 1, 2, 3, \dots \}$. Então 

$$
\sum\limits_{p=0}^n (-1)^{p-n}(-1)^p(-1)^{n-p} \binom{n}{p}
$$ 

vale:

\begin{multicols}{5}
    \begin{enumerate}[\bf A (\quad)]
        \item -1
        \item 0
        \item 1
        \item 2
        \item n.d.a.
    \end{enumerate}
\end{multicols}

\textbf{Questão 12.} A desigualdade $a^3 + 1/a^3 > a^2 + 1/a^2$ é verdadeira se:

\begin{multicols}{3}
    \begin{enumerate}[\bf A (\quad)]
        \item $|a| > 1$
        \item $a \neq 1$, $a \neq 0$
        \item $a > 0$ e $a \neq 1$
        \item $|a| < 1$, $a \neq 0$
        \item n.d.a.
    \end{enumerate}
\end{multicols}


\textbf{Questão 13.} Entre 4 e 5 horas o ponteiro das horas de um relógio fica  duas  vezes  em  ângulo  reto  com  o  ponteiro  dos  minutos. Os momentos destas ocorrências serão:

\begin{enumerate}[\bf A (\quad)]
    \item 4h5$\dfrac{2}{11}$min e 4h38$\dfrac{5}{11}$min
    \item 4h5$\dfrac{5}{11}$min e 4h38$\dfrac{2}{11}$min
    \item 4h5$\dfrac{5}{11}$min e 4h38$\dfrac{5}{12}$min
    \item 4h5$\dfrac{3}{11}$min e 4h38$\dfrac{7}{11}$min
    \item n.d.a.
\end{enumerate}


\textbf{Questão 14.} Seja a equação do $4^{\circ}$ grau 
$$
x^4 + qx^3 + rx^2 + sx + t = 0 
$$
onde $q$, $r$, $s$ e $t$ são números racionais não nulos tais que: $L$, $M$, $N$ e $P$ são raízes reais dessa equação. O valor de $\dfrac{L}{MNP} + \dfrac{M}{LNP} + \dfrac{N}{LMP} + \dfrac{P}{LMN}$ é:

\begin{multicols}{3}
    \begin{enumerate}[\bf A (\quad)]
        \item $\dfrac{(q^2 - 2r)}{t}$
        \item $\dfrac{(q^2 - r + s)}{t}$
        \item $\dfrac{(q^2 - r)}{t}$
        \item $\dfrac{q}{r} + \dfrac{r}{s} + \dfrac{s}{t} + \dfrac{t}{q}$
        \item n.d.a.
    \end{enumerate}
\end{multicols}

\textbf{Questão 15.} Um  octaedro  regular  é  inscrito  num  cubo,  que  está  inscrito  numa  esfera,  e  que  está  inscrita  num  tetraedro  regular. Se o comprimento da aresta do tetraedro é 1, qual é o comprimento da aresta do octaedro? 

\begin{multicols}{5}
    \begin{enumerate}[\bf A (\quad)]
        \item $\sqrt{\dfrac{2}{27}}$
        \item $\dfrac{\sqrt{3}}{4}$
        \item $\dfrac{\sqrt{2}}{4}$
        \item $\dfrac{1}{6}$
        \item n.d.a.
    \end{enumerate}
\end{multicols}

\textbf{Questão 16.} Certa  liga  contém  20\%  de  cobre  e  5\%  de  estanho.  Quantos  quilos  de  cobre  e  quantos  quilos  de  estanho  devem  ser  adicionados  a  100  quilos  dessa  liga  para  a  obtenção  de  uma  outra  com  30\%  de  cobre  e  10\%  de  estanho? (Todas as percentagens em kg). 

\begin{enumerate}[\bf A (\quad)]
    \item 8 kg de cobre e 6 kg de estanho
    \item 17,50 kg de cobre e 7,5 kg e estanho
    \item 18 kg de cobre e 7,5 kg de estanho
    \item 17,50 kg de cobre e 7,8 kg de estanho.
    \item n.d.a.
\end{enumerate}

\textbf{Questão 17.} A  lei  de  decomposição  do  radium  no  tempo  $t  \geq  0$  ,  é  dada por $M(t)=Ce^{-kt}$, onde $M(t)$ é a quantidade de radium no tempo $t$, $C$ e $k$ são constantes positivas e e é a base do logaritmo neperiano. Se a metade da quantidade primitiva $M(0)$,   desaparece   em   1600   anos,   qual   a   quantidade   perdida em 100 anos?


\begin{enumerate}[\bf A (\quad)]
    \item $1 - 100^{-1}$ da quantidade inicial
    \item $1 - 2^{-6}$ da quantidade inicial
    \item $1 - 2^{-16}$ da quantidade inicial
    \item $1 - 2^{-1/16}$ da quantidade inicial
    \item n.d.a.
\end{enumerate}

\textbf{Questão 18.} Seja a equação: 

$$
(\log_e m)\sin x \pm \cos x = \log_e m
$$

Quais  as  condições  sobre  $m$  para  que  a  equação  dada  admita solução?


\begin{enumerate}[\bf A (\quad)]
    \item $m > 0$ se $x = (2k + \frac{1}{2})\pi$; $m > 0$ e $m \neq 1$ se $x \neq (2k + \frac{1}{2})\pi$
    \item $m \neq 0$ se $x = (2k + \frac{1}{2})\pi$; $m \geq e$ e $m \neq 1$ se $x \neq (2k + \frac{1}{2})\pi$
    \item $m > e$ se $x = (2k + \frac{1}{2})\pi$; $m \geq 1$ se $x \neq (2k + \frac{1}{2})\pi$
    \item $m > -1/e$ e $m \neq 0$ se $x = (2k + \frac{1}{2})\pi$; $m \neq 0$ se $x \neq (2k + \frac{1}{2})\pi$
    \item n.d.a.
\end{enumerate}

\textbf{Questão 19.} Eliminando $\theta$ no sistema de equações $(a > 0)$, temos: 

$$
\begin{cases}
x \sin \theta + y \cos \theta = 2a \sin \theta\\
x \cos \theta - y \sin \theta = a \cos \theta
\end{cases}
$$

\begin{enumerate}[\bf A (\quad)]
    \item $(x + y)^{\frac{2}{3}} - (x - y)^{\frac{2}{3}} = 2a(x + y)^2$
    \item $(x + y)^2 + (x - y)^2 = (x + y)a$
    \item $(x + y)^{\frac{2}{3}} + (x - y)^{\frac{2}{3}} = 2a^{\frac{2}{3}}$
    \item impossível eliminar $\theta$
    \item n.d.a.
\end{enumerate}

\textbf{Questão 20.} Um  cliente  deposita  num  fundo  de  investimento  Cr\$  1.000,00 anualmente, durante 5 anos. Seu capital, no final de  cada  ano,  é  acrescido  de  10\%.  No  final  de  5  anos  seu  capital acumulado será Cr\$:

\begin{enumerate}[\bf A (\quad)]
    \item 6.715,00
    \item 6.715,62
    \item 6.715,60
    \item 6.715,61
    \item n.d.a.
\end{enumerate}

\textbf{Questão 21.} Durante o eclipse total do sol de 07 de março de 1970 a  largura  da  faixa  da  escuridão  total  foi  de  100  km.  Em  cada  ponto  do  eixo  central  desta  faixa,  a  duração  do  período  de  escuridão  total  foi  de  3  minutos.  Qual  foi  a  duração  deste  período  num  ponto  situado  a  10  km  do  limite da faixa de escuridão total?

\begin{multicols}{3}
    \begin{enumerate}[\bf A (\quad)]
        \item 1 min 36 seg
        \item 1 min 48 seg
        \item 1 min 30 seg
        \item 0 min 36 seg
        \item n.d.a.
    \end{enumerate}
\end{multicols}

\textbf{Questão 22.} Seja a equação: 

$$
3 \tan 3x = [3 (\log_e t)^2 - 4 \log_e t + 2] \tan x, \, x \neq n\pi
$$

Quais  as  condições  sobre  $t$  para  que  a  equação  acima  admita solução?

\begin{enumerate}[\bf A (\quad)]
    \item $0 < t < 1/e$ ou $e^{1/3} < t < e$ ou $t > e^{7/3}$
    \item $e^{1/3} \leq t \leq e^{3/2}$ ou $0 < t < e$
    \item $e^{1/3} < t \leq e^{2/3}$ ou $1/e > t$
    \item $t > 0$ e $t \neq 1$
    \item n.d.a.
\end{enumerate}


\textbf{Questão 23.} Seja  $L$  o  comprimento  do  eixo  de  uma  caldeira  cilíndrica terminada por duas semi-esferas. Sabe-se que a área  da  superfície  total  da  caldeira  é  $4\pi k^2$,  com  $0  <  k  <  L/2$. As dimensões da parte cilíndrica da caldeira valem: 

\begin{multicols}{3}
    \begin{enumerate}[\bf A (\quad)]
        \item $ k^2/L$ e $L + 3k^2/L $
        \item $ k^2/L$ e $k + (3/4)L$
        \item $2k^2 / L$ e $L - 4k^2 / L$
        \item $ k^2/2L$ e $L + (4/2)k^2$
        \item n.d.a.
    \end{enumerate}
\end{multicols}

\textbf{Questão 24.} Seja  $S$  uma  semi-esfera  de  raio  $R$  dado.  Sejam  $p$  e  $q$  dois  planos  paralelos  e  distantes  entre  si  $R/2$  e  tais  que  interceptem  $S$  paralelamente  à  sua  base.  Seja  $T$  o  tronco  de cone com bases $b$ e $c$, onde $b$ e são as interseções de $p$ e $q$ com $S$. Seja $x$ o valor da menor das distâncias $d$ e $D$, onde  $d$  é  a  distância  entre  $p$  e  a  base  de  $S$,  e  $D$  é  a  distância  entre  $q$  e  a  base  de  $S$.  Seja  $k$  

$$
k = \left\{ \left( R^2 - x^2 \right) \left[ R^2 - \left( x + \frac{R}{2} \right)^2 \right] \right\}^{\frac{1}{2}}
$$
%
Então  o  volume  de  $T$,  como  função  de  $x$,  $0 \leq x \leq R/2$ vale:  

\begin{enumerate}[\bf A (\quad)]
    \item $\dfrac{\pi R}{6} \left( \dfrac{7}{4}R^2 - 2x^2 - Rx + k \right)$
    \item $\dfrac{\pi R}{12} \left( \dfrac{7}{4}R^2 - 2x^2 - Rx + k \right)$
    \item $\dfrac{\pi R}{12} \left( \dfrac{7}{4}R^2 - 2x^2 - Rx - k \right)$
    \item $\dfrac{\pi R}{6} \left( \dfrac{7}{4}R^2 - 2x^2 - Rx - k \right)$
    \item n.d.a.
\end{enumerate}



\textbf{Questão 25.} A  solução  da  equação 

$$
\log_u \left[ \sum\limits_{k=1}^{n} \left( \frac{k}{2(k+1)!} \right) \right].x = 1
$$

com $u = 1/(n+2)!$, é  

\begin{multicols}{3}
    \begin{enumerate}[\bf A (\quad)]
        \item $\dfrac{2}{[(n + 1)! - 1]}$
        \item $\dfrac{2}{[n(n + 1)! - 1]}$
        \item $\dfrac{2}{[(n + 2)! - (n + 2)]}$
        \item $\dfrac{[(n + 1)! - 1]}{2n}$
        \item n.d.a.
    \end{enumerate}
\end{multicols}

\end{document}