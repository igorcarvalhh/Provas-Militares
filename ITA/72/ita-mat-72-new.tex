\documentclass[11pt]{article}
\usepackage{enumerate}
\usepackage{amsmath}
\usepackage{blindtext}
\usepackage{multicol}
\usepackage{amsfonts}
\usepackage[left=1.5cm,top=2cm,right=1.5cm,bottom=2cm]{geometry}

\newcommand*\varhrulefill[1][0.4pt]{\leavevmode\leaders\hrule height#1\hfill\kern0pt}
\begin{document}

\begin{center}
    \textbf{MATEMÁTICA}
\end{center}

\noindent\varhrulefill[0.4mm]

\vspace{6pt}

\noindent \textbf{Notações}

\vspace{6pt}

$\mathbb{N}\quad = \{ 1,2,3, \dots \}$: o conjunto dos números naturais.

$\mathbb{R}\quad$: o conjunto dos números reais.

$\mathbb{C}\quad$: o conjunto dos números complexos.

$i \, \, \quad$: unidade imaginária, $i^2 = -1$.

\vspace{6pt}

\noindent Observação: Os sistemas de coordenadas considerados são os cartesianos retangulares.

\noindent\varhrulefill[0.4mm]

\vspace{6pt}

\parindent0em

\textbf{Questão 1.} O ângulo convexo formado pelos ponteiros das horas e dos minutos às 10 horas e 15 minutos é: 

\begin{multicols}{5}
    \begin{enumerate}[\bf A (\quad)]
        \item $142^{\circ}30$' 
        \item $142^{\circ}40$’
        \item $142^{\circ}$
        \item $141^{\circ}30$’
        \item n.d.a.
    \end{enumerate}
\end{multicols}

\textbf{Questão 2.} Todas as raízes reais da equação são:

$$
\sqrt{\dfrac{x^2 + 3}{x}} - \sqrt{\dfrac{x}{x^2 + 3}} = \dfrac{3}{2}
$$

\begin{enumerate}[\bf A (\quad)]
    \item $x_1 = 3$ e $x_2 = -3$
    \item $x_1 = 3$ e $x_2 = 3$
    \item $x_1 = 3$ e $x_2 = 3$
    \item não tem raízes reais
    \item n.d.a.
\end{enumerate}

\textbf{Questão 3.} Todas as raízes reais da equação são

$$
x^{-1} + 4x^{-\frac{1}{2}} + 3 = 0
$$

\begin{multicols}{3}
    \begin{enumerate}[\bf A (\quad)]
        \item $x_1 = 1$ e $x_2 = 1$
        \item $x_1 = 1/3$ e $x_2 = 1/3$
        \item $x_1 = 3$ e $x_2 = 3$
        \item não tem raízes reais
        \item n.d.a.
    \end{enumerate}
\end{multicols}

\textbf{Questão 4.} Qual é a relação que $a$, $b$ e $c$ devem satisfazer tal que o sistema abaixo tenha pelo menos uma solução?

$$
\begin{cases}
x + 2y -3z = a\\
2x + 6y - 11z = b \\
x + 2y + 7z = c
\end{cases}
$$

\begin{enumerate}[\bf A (\quad)]
    \item $5a = 2b - c$
    \item $5a = 2b + c$
    \item $5a \neq 2b + c$
    \item não existe relação entre $a$, $b$ e $c$
    \item n.d.a.
\end{enumerate}

\textbf{Questão 5.} Assinale a sentença correta.

\begin{enumerate}[\bf A (\quad)]
    \item $a > 1$ e $\log_a x < 0$ se $x > 1$, $\log_a x > 0$ se $x < 1$
    \item $0 < a < 1$ e $\log_a x > 0$ se $x < 1$, $\log_a x < 0$ se $x > 1$
    \item $a > 1$ e $\log_a x_1 < \log_a x_2$ se, e só se, $x_1 > x_2$
    \item $0 < a < 1$ e $\log_a x_1 > \log_a x_2$ se, e só se, $x_1 < x_2$
    \item n.d.a.
\end{enumerate}

\textbf{Questão 6.} Assinale  uma  solução  para  a  equação  trigonométrica 

$$
\sqrt{3} \sin x + \cos x = \sqrt{3}
$$

\begin{multicols}{3}
    \begin{enumerate}[\bf A (\quad)]
        \item $x = 2k\pi - \pi/6$
        \item $x = 2k\pi + \pi/6$
        \item $x = 2k\pi + \pi/2$
        \item $x = 2k\pi - \pi/2$
        \item n.d.a.
    \end{enumerate}
\end{multicols}

\textbf{Questão 7.} Qual é o valor de m para que $\dfrac{C_m^3}{C_{m-1}^3} = \dfrac{7}{4}$? 



\begin{enumerate}[\bf A (\quad)]
    \item $m = 8$
    \item $m = 10$
    \item $m = 6$
    \item $m = 5$
    \item n.d.a.
\end{enumerate}

\textbf{Questão 8.} Consideremos   duas   retas   $r_1$   e   $r_2$   ortogonais   não   situadas   num   mesmo   plano,   e   um   segmento   $XY$   de   comprimento  constante  que  desliza  suas  extremidades  sobre essas retas. O lugar geométrico, das interseções dos planos construídos perpendicularmente a essas retas $r_1$ e $r_2$ nas extremidades do segmento $XY$, é:

\begin{enumerate}[\bf A (\quad)]
    \item uma reta perpendicular ao segmento $XY$
    \item a superfície cilíndrica de revolução tendo como diretriz a parábola.
    \item a superfície cilíndrica de revolução tendo como diretriz a elipse.
    \item a superfície cilíndrica de revolução tendo como diretriz a hipérbole.
    \item n.d.a.
\end{enumerate}

\textbf{Questão 9.} Dado  um  cilindro  de  revolução  de  raio  $r$  e  altura  $h$;  sabendo-se que a média harmônica entre o raio $r$ e a altura $h$ é 4 e que sua área total é $2\pi$ u.a. O raio r deve satisfazer a relação: 

\begin{multicols}{3}
    \begin{enumerate}[\bf A (\quad)]
        \item $r^3 - r + 2 = 0$
        \item $r^3 - 4r^2 + 5r - 2 = 0$
        \item $r^3 - r^2 - r + 1 = 0$
        \item $r^3 - 3r -3 = 0$
        \item n.d.a.
    \end{enumerate}
\end{multicols}

\textbf{Questão 10.} Seja  $B`C`$  a  projeção  do  diâmetro  $BC$  de  um  circulo  de  raio  $r$  sobre  a  reta  tangente  $t$  por  um  ponto  $M$  deste  círculo.  Seja  $2k$  a  razão  da  área  total  do  tronco  do  cone  gerado  pela  rotação  do  trapézio  $BCB`C`$  ao  redor  da  reta  tangente  $t$  e  a  área  do  círculo  dado.  Qual  é  o  valor  de  $k$  para que a medida do segmento $MB$ seja igual a metade do raio $r$? 

\begin{multicols}{3}
    \begin{enumerate}[\bf A (\quad)]
        \item $k = 11/3$
        \item $k = 15/4$
        \item $k = 2$
        \item $k = 1/2$
        \item n.d.a.
    \end{enumerate}
\end{multicols}

\textbf{Questão 11.} Seja a equação:

$$
3^{(\ln x) + 1} - 3^{(\ln x) - 1} + 3^{(\ln x) - 3} - 3^{(\ln x) - 4} = \log_e \dfrac{\sin a}{e^{-657}}
$$

Sabe-se   que   $\ln x$   é   igual   a   menor   raiz   da   equação   $r^2 - 4r - 5 = 0$.  O  valor  de  a  para  que  a  equação  seja  verificada é: 

\begin{multicols}{3}
    \begin{enumerate}[\bf A (\quad)]
        \item $a = 3\pi/2$
        \item $a = \arcsin (\sqrt{2}/2)$
        \item $a = \arcsin (1/e^3)$
        \item $a = \arcsin (e)$
        \item n.d.a.
    \end{enumerate}
\end{multicols}

\textbf{Questão 12.} Quais  os  valores  de  a  de  modo  que  o  sistema  admita  soluções não triviais?

$$
\begin{cases}
(\sin{\alpha} - 1)x + 2y -(\sin{\alpha})z = 0\\
(3\sin{\alpha})y + 4z = 0 \\
3x + (7\sin{\alpha})y + 6z = 0
\end{cases}
$$

\begin{enumerate}[\bf A (\quad)]
    \item $\alpha = n\pi$, $n = 0,\pm 1,\pm 2, \pm 3, \dots$
    \item $\alpha = n\pi + \pi/3$, $n = 0,\pm 1,\pm 2, \pm 3, \dots$
    \item $\alpha = n\pi + \pi/2$, $n = 0,\pm 1,\pm 2, \pm 3, \dots$
    \item não há valores de $\alpha$
    \item n.d.a.
\end{enumerate}


\textbf{Questão 13.} As  dimensões  de  um  paralelepípedo  retângulo  estão  em progressão geométrica e a sua soma vale s. Sabendo-se  que  o  seu  volume  é  $v^3$,  $s \geq 3v$,  então  duas  de  suas  dimensões são:

\begin{enumerate}[\bf A (\quad)]
    \item $\dfrac{s + v \pm \sqrt{(s + v)^2 - v^2}}{2}$
    \item $s - v$ e $v + s$
    \item $v \pm \sqrt{(s - v)^2 - 4v^2}$
    \item $\dfrac{s - v \pm \sqrt{(s + v)^2 - 4v^2}}{2}$
    \item n.d.a.
\end{enumerate}


\textbf{Questão 14.} Construindo-se um prisma e uma pirâmide sobre uma mesma  base  de  área  $A$  e  volumes  $V_1$  e  $V_2$,  a  área  da  secção da pirâmide com a outra base do prisma é: 

\begin{enumerate}[\bf A (\quad)]
    \item $A\dfrac{V_1}{V_1 + V_2}$
    \item $\dfrac{V_2 - V_1}{AV_2}$
    \item $A \left( 1 - \dfrac{V_1}{3V_2} \right)$
    \item $A \dfrac{3V_2 - V_1}{V_2}$
    \item n.d.a.
\end{enumerate}

\textbf{Questão 15.} Para todo $\alpha$ e $\beta$, $|\beta| < 1$, a expressão abaixo é igual a:

$$
\tan (\arctan \alpha + \arcsin \beta)
$$


\begin{enumerate}[\bf A (\quad)]
    \item $-\dfrac{\beta + \alpha \sqrt{1-\beta^2}}{\alpha \beta - \sqrt{1 - \beta^2}}$
    \item $\dfrac{\alpha - \beta}{\alpha \beta + \sqrt{1 - \beta^2}}$
    \item $\dfrac{\alpha - \beta}{\alpha \beta \sqrt{1 - \beta^2} - 1}$
    \item $\dfrac{\sqrt{1-\beta^2}(\alpha - \beta)}{\alpha \beta - 1}$
    \item n.d.a.
\end{enumerate}

\textbf{Questão 16.} A soma dos quadrados das raízes da equação $2x^3 - 8x^2 - 60x + k = 0$ ($k$ constante) é:

\begin{multicols}{3}
    \begin{enumerate}[\bf A (\quad)]
        \item $76 + k^2$
        \item $(34 + k)^2$
        \item $66$
        \item $76$
        \item n.d.a.
    \end{enumerate}
\end{multicols}

\textbf{Questão 17.} Seja  $f(x)  =  x^2  +  px  +  p$  uma  função  real  de  variável  real.  Os  valores  de  $p$  para  os  quais  $f(x)  =  0$  possua  raiz  dupla positiva, são:

\begin{enumerate}[\bf A (\quad)]
    \item $0 < p < 4$
    \item $p = 4$
    \item $p = 0$
    \item $f(x) = 0$ não pode ter raiz dupla positiva
    \item n.d.a.
\end{enumerate}

\textbf{Questão 18.} O volume do sólido gerado por um triângulo, que gira em torno de sua hipotenusa cujos catetos são 15 cm e 20 cm, é:


\begin{enumerate}[\bf A (\quad)]
    \item $1080\pi \, cm^3$
    \item $960\pi \, cm^3$
    \item $1400\pi \, cm^3$
    \item $1600\pi \, cm^3$
    \item n.d.a.
\end{enumerate}

\textbf{Questão 19.} Seja a equação: 

$$
3 \tan 3x = [3(\log k)^2 - 4\log k + 2] \tan x
$$

Para  que  intervalo  de  valores  de  $k$;  abaixo,  a  equação  dada admite solução?

\begin{multicols}{3}
    \begin{enumerate}[\bf A (\quad)]
        \item $0 < k \leq e^{1/3}$
        \item $0 < k \leq e^{2/3}$
        \item $0 < k \leq 1/e$
        \item $0 < k \leq e^{7/3}$
        \item n.d.a.
    \end{enumerate}
\end{multicols}

\textbf{Questão 20.} Seja a equação $P(x) = 0$, onde $P(x)$ é um polinômio de coeficientes inteiros.   Se   $P(x)$   admite   uma   raiz   inteira,   então   $P(-1).P(0).P(1)$ necessariamente: 

\begin{enumerate}[\bf A (\quad)]
    \item vale 5
    \item vale 3
    \item é divisível por 5
    \item é divisível por 3
    \item n.d.a.
\end{enumerate}

\textbf{Questão 21.} Seja $A$ um conjunto finito com m elementos e $In = \{1, 2,  ...,  n\}$.  O  número  de  todas  as  funções  definidas  em  $In$ com valores em $A$ é:

\begin{multicols}{3}
    \begin{enumerate}[\bf A (\quad)]
        \item $C_m^n$
        \item $mn$
        \item $n^m$
        \item $m^n$
        \item n.d.a.
    \end{enumerate}
\end{multicols}

\textbf{Questão 22.} Sejam $n \leq m , Im = \{1, 2, ..., m\}$ e $In =\{1, 2, ..., n\}$. O número   de   funções   biunívocas   definidas   em   $Im$   com   valores em $In$ é: 

\begin{enumerate}[\bf A (\quad)]
    \item $A_m^n$
    \item $C_m^n$
    \item $m!/n!$
    \item $mn$
    \item n.d.a.
\end{enumerate}


\textbf{Questão 23.} Seja $\theta = \arcsin (b/a)$, com $|a| > |b|$. Então $2\theta$ vale:

\begin{multicols}{3}
    \begin{enumerate}[\bf A (\quad)]
        \item $\arcsin \left( \dfrac{2a}{b} \right)$
        \item $\arcsin \left( \dfrac{2b}{a} \right)$
        \item $\arcsin \left( \dfrac{2a}{\sqrt{b^2 - a^2}} \right)$
        \item $\arcsin \left( \dfrac{2b}{a^2} \sqrt{a^2 - b^2} \right)$
        \item n.d.a.
    \end{enumerate}
\end{multicols}

\textbf{Questão 24.} Quais  condições  devem  satisfazer  $a$  e  $k$  para  que  a  seguinte igualdade: $\log (\sec a) = k$ tenha sentido? 

\begin{multicols}{3}
    \begin{enumerate}[\bf A (\quad)]
        \item $-\pi/2 < a < \pi/2$, $k \geq 0$
        \item $-\pi/2 < a < \pi/2$, $k < 0$
        \item $-\pi/2 < a \leq \pi/2$, $k > 0$
        \item $-\pi/2 < a < 3\pi/2$, $k \geq 0$
        \item n.d.a.
    \end{enumerate}
\end{multicols}

\textbf{Questão 25.} Consideremos  a  função  $S(x) = \sum\limits_{n=1}^{\infty} (\sin x)^n$,  onde  $0 < x < \pi/2$. Para que valores de $x$; $10 \leq S(x) \leq 20$? 


\begin{enumerate}[\bf A (\quad)]
    \item $\arcsin (9/10) \leq x \leq \arcsin (19/20)$
    \item $\arcsin (10/9) \leq x \leq \arcsin (20/19)$
    \item $\arcsin (10/11) \leq x \leq \arcsin (\sqrt{3}/2)$
    \item $\arcsin (\sqrt{2}/2) \leq x \leq \arcsin (\sqrt{3}/2)$
    \item n.d.a.
\end{enumerate}

\end{document}